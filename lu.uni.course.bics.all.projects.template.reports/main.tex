\documentclass[conference,compsoc]{IEEEtran}
\usepackage{graphicx}
\usepackage[table]{xcolor}

\usepackage{datetime}

% *** CITATION PACKAGES ***
%
\ifCLASSOPTIONcompsoc
  % IEEE Computer Society needs nocompress option
  % requires cite.sty v4.0 or later (November 2003)
  \usepackage[nocompress]{cite}
\else
  % normal IEEE
  \usepackage{cite} 
\fi 
% cite.sty was written by Donald Arseneau
% V1.6 and later of IEEEtran pre-defines the format of the cite.sty package
% \cite{} output to follow that of the IEEE. Loading the cite package will
% result in citation numbers being automatically sorted and properly
% "compressed/ranged". e.g., [1], [9], [2], [7], [5], [6] without using
% cite.sty will become [1], [2], [5]--[7], [9] using cite.sty. cite.sty's
% \cite will automatically add leading space, if needed. Use cite.sty's
% noadjust option (cite.sty V3.8 and later) if you want to turn this off
% such as if a citation ever needs to be enclosed in parenthesis.
% cite.sty is already installed on most LaTeX systems. Be sure and use
% version 5.0 (2009-03-20) and later if using hyperref.sty.
% The latest version can be obtained at:
% http://www.ctan.org/pkg/cite
% The documentation is contained in the cite.sty file itself.
%
% Note that some packages require special options to format as the Computer
% Society requires. In particular, Computer Society  papers do not use
% compressed citation ranges as is done in typical IEEE papers
% (e.g., [1]-[4]). Instead, they list every citation separately in order
% (e.g., [1], [2], [3], [4]). To get the latter we need to load the cite
% package with the nocompress option which is supported by cite.sty v4.0
% and later.

% *** GRAPHICS RELATED PACKAGES ***
%
\ifCLASSINFOpdf
  % \usepackage[pdftex]{graphicx}
  % declare the path(s) where your graphic files are
  % \graphicspath{{../pdf/}{../jpeg/}}
  % and their extensions so you won't have to specify these with
  % every instance of \includegraphics
  % \DeclareGraphicsExtensions{.pdf,.jpeg,.png}
\else
  % or other class option (dvipsone, dvipdf, if not using dvips). graphicx
  % will default to the driver specified in the system graphics.cfg if no
  % driver is specified.
  % \usepackage[dvips]{graphicx}
  % declare the path(s) where your graphic files are
  % \graphicspath{{../eps/}}
  % and their extensions so you won't have to specify these with
  % every instance of \includegraphics
  % \DeclareGraphicsExtensions{.eps}
\fi
% graphicx was written by David Carlisle and Sebastian Rahtz. It is
% required if you want graphics, photos, etc. graphicx.sty is already
% installed on most LaTeX systems. The latest version and documentation
% can be obtained at: 
% http://www.ctan.org/pkg/graphicx
% Another good source of documentation is "Using Imported Graphics in
% LaTeX2e" by Keith Reckdahl which can be found at:
% http://www.ctan.org/pkg/epslatex
%
% latex, and pdflatex in dvi mode, support graphics in encapsulated
% postscript (.eps) format. pdflatex in pdf mode supports graphics
% in .pdf, .jpeg, .png and .mps (metapost) formats. Users should ensure
% that all non-photo figures use a vector format (.eps, .pdf, .mps) and
% not a bitmapped formats (.jpeg, .png). The IEEE frowns on bitmapped formats
% which can result in "jaggedy"/blurry rendering of lines and letters as
% well as large increases in file sizes.
%
% You can find documentation about the pdfTeX application at:
% http://www.tug.org/applications/pdftex


% *** MATH PACKAGES ***
%
%\usepackage{amsmath}
% A popular package from the American Mathematical Society that provides
% many useful and powerful commands for dealing with mathematics.
%
% Note that the amsmath package sets \interdisplaylinepenalty to 10000
% thus preventing page breaks from occurring within multiline equations. Use:
%\interdisplaylinepenalty=2500
% after loading amsmath to restore such page breaks as IEEEtran.cls normally
% does. amsmath.sty is already installed on most LaTeX systems. The latest
% version and documentation can be obtained at:
% http://www.ctan.org/pkg/amsmath

% *** SPECIALIZED LIST PACKAGES ***
%
%\usepackage{algorithmic}
% algorithmic.sty was written by Peter Williams and Rogerio Brito.
% This package provides an algorithmic environment fo describing algorithms.
% You can use the algorithmic environment in-text or within a figure
% environment to provide for a floating algorithm. Do NOT use the algorithm
% floating environment provided by algorithm.sty (by the same authors) or
% algorithm2e.sty (by Christophe Fiorio) as the IEEE does not use dedicated
% algorithm float types and packages that provide these will not provide
% correct IEEE style captions. The latest version and documentation of
% algorithmic.sty can be obtained at:
% http://www.ctan.org/pkg/algorithms
% Also of interest may be the (relatively newer and more customizable)
% algorithmicx.sty package by Szasz Janos:
% http://www.ctan.org/pkg/algorithmicx


% *** ALIGNMENT PACKAGES ***
%
%\usepackage{array}
% Frank Mittelbach's and David Carlisle's array.sty patches and improves
% the standard LaTeX2e array and tabular environments to provide better
% appearance and additional user controls. As the default LaTeX2e table
% generation code is lacking to the point of almost being broken with
% respect to the quality of the end results, all users are strongly
% advised to use an enhanced (at the very least that provided by array.sty)
% set of table tools. array.sty is already installed on most systems. The
% latest version and documentation can be obtained at:
% http://www.ctan.org/pkg/array

% IEEEtran contains the IEEEeqnarray family of commands that can be used to
% generate multiline equations as well as matrices, tables, etc., of high
% quality.

% *** SUBFIGURE PACKAGES ***
%\ifCLASSOPTIONcompsoc
%  \usepackage[caption=false,font=footnotesize,labelfont=sf,textfont=sf]{subfig}
%\else
%  \usepackage[caption=false,font=footnotesize]{subfig}
%\fi
% subfig.sty, written by Steven Douglas Cochran, is the modern replacement
% for subfigure.sty, the latter of which is no longer maintained and is
% incompatible with some LaTeX packages including fixltx2e. However,
% subfig.sty requires and automatically loads Axel Sommerfeldt's caption.sty
% which will override IEEEtran.cls' handling of captions and this will result
% in non-IEEE style figure/table captions. To prevent this problem, be sure
% and invoke subfig.sty's "caption=false" package option (available since
% subfig.sty version 1.3, 2005/06/28) as this is will preserve IEEEtran.cls
% handling of captions.
% Note that the Computer Society format requires a sans serif font rather
% than the serif font used in traditional IEEE formatting and thus the need
% to invoke different subfig.sty package options depending on whether
% compsoc mode has been enabled.
%
% The latest version and documentation of subfig.sty can be obtained at:
% http://www.ctan.org/pkg/subfig

% *** FLOAT PACKAGES ***
%
%\usepackage{fixltx2e}
% fixltx2e, the successor to the earlier fix2col.sty, was written by
% Frank Mittelbach and David Carlisle. This package corrects a few problems
% in the LaTeX2e kernel, the most notable of which is that in current
% LaTeX2e releases, the ordering of single and double column floats is not
% guaranteed to be preserved. Thus, an unpatched LaTeX2e can allow a
% single column figure to be placed prior to an earlier double column
% figure.
% Be aware that LaTeX2e kernels dated 2015 and later have fixltx2e.sty's
% corrections already built into the system in which case a warning will
% be issued if an attempt is made to load fixltx2e.sty as it is no longer
% needed.
% The latest version and documentation can be found at:
% http://www.ctan.org/pkg/fixltx2e

%\usepackage{stfloats}
% stfloats.sty was written by Sigitas Tolusis. This package gives LaTeX2e
% the ability to do double column floats at the bottom of the page as well
% as the top. (e.g., "\begin{figure*}[!b]" is not normally possible in
% LaTeX2e). It also provides a command:
%\fnbelowfloat
% to enable the placement of footnotes below bottom floats (the standard
% LaTeX2e kernel puts them above bottom floats). This is an invasive package
% which rewrites many portions of the LaTeX2e float routines. It may not work
% with other packages that modify the LaTeX2e float routines. The latest
% version and documentation can be obtained at:
% http://www.ctan.org/pkg/stfloats
% Do not use the stfloats baselinefloat ability as the IEEE does not allow
% \baselineskip to stretch. Authors submitting work to the IEEE should note
% that the IEEE rarely uses double column equations and that authors should try
% to avoid such use. Do not be tempted to use the cuted.sty or midfloat.sty
% packages (also by Sigitas Tolusis) as the IEEE does not format its papers in
% such ways.
% Do not attempt to use stfloats with fixltx2e as they are incompatible.
% Instead, use Morten Hogholm'a dblfloatfix which combines the features
% of both fixltx2e and stfloats:
%
% \usepackage{dblfloatfix}
% The latest version can be found at:
% http://www.ctan.org/pkg/dblfloatfix

% *** PDF, URL AND HYPERLINK PACKAGES ***
%
%\usepackage{url}
% url.sty was written by Donald Arseneau. It provides better support for
% handling and breaking URLs. url.sty is already installed on most LaTeX
% systems. The latest version and documentation can be obtained at:
% http://www.ctan.org/pkg/url
% Basically, \url{my_url_here}.

% *** Do not adjust lengths that control margins, column widths, etc. ***
% *** Do not use packages that alter fonts (such as pslatex).         ***
% There should be no need to do such things with IEEEtran.cls V1.6 and later.
% (Unless specifically asked to do so by the journal or conference you plan
% to submit to, of course. )

% correct bad hyphenation here
\hyphenation{op-tical net-works semi-conduc-tor}
   
\usepackage{hyperref}
 
\begin{document}
% 
% paper title
% Titles are generally capitalized except for words such as a, an, and, as,
% at, but, by, for, in, nor, of, on, or, the, to and up, which are usually
% not capitalized unless they are the first or last word of the title.
% Linebreaks \\ can be used within to get better formatting as desired.
% Do not put math or special symbols in the title.
\title{BSPro - A First Bachelor Semester Project in BiCS-land\\
{\small \today~-~\currenttime}}

 
% author names and affiliations
% use a multiple column layout for up to three different
% affiliations
\author{\IEEEauthorblockN{Dedic Adel}
\IEEEauthorblockA{University of Luxembourg\\
Email: adel.dedic.001@student.uni.lu}
\\
{\bf This report has been produced under the supervision of:}\\
\IEEEauthorblockN{Benoit Ries}
\IEEEauthorblockA{University of Luxembourg\\
Email: benoit.ries@uni.lu}%
}

% conference papers do not typically use \thanks and this command
% is locked out in conference mode. If really needed, such as for
% the acknowledgment of grants, issue a \IEEEoverridecommandlockouts
% after \documentclass

% for over three affiliations, or if they all won't fit within the width
% of the page (and note that there is less available width in this regard for
% compsoc conferences compared to traditional conferences), use this
% alternative format:
% 
%\author{\IEEEauthorblockN{Michael Shell\IEEEauthorrefmark{1},
%Homer Simpson\IEEEauthorrefmark{2},
%James Kirk\IEEEauthorrefmark{3}, 
%Montgomery Scott\IEEEauthorrefmark{3} and
%Eldon Tyrell\IEEEauthorrefmark{4}}
%\IEEEauthorblockA{\IEEEauthorrefmark{1}School of Electrical and Computer Engineering\\
%Georgia Institute of Technology,
%Atlanta, Georgia 30332--0250\\ Email: see http://www.michaelshell.org/contact.html}
%\IEEEauthorblockA{\IEEEauthorrefmark{2}Twentieth Century Fox, Springfield, USA\\
%Email: homer@thesimpsons.com}
%\IEEEauthorblockA{\IEEEauthorrefmark{3}Starfleet Academy, San Francisco, California 96678-2391\\
%Telephone: (800) 555--1212, Fax: (888) 555--1212}
%\IEEEauthorblockA{\IEEEauthorrefmark{4}Tyrell Inc., 123 Replicant Street, Los Angeles, California 90210--4321}}




% use for special paper notices
%\IEEEspecialpapernotice{(Invited Paper)}




% make the title area
\maketitle

%to remove for your report
%\footnote{}

% As a general rule, do not put math, special symbols or citations
% in the abstract
\begin{abstract}
This document is a template for the scientific and technical (S\&T for short) report that is to be delivered by any BiCS student at the end of each Bachelor Semester Project (BSP). The Latex source files are available at:\\
\href{https://github.com/nicolasguelfi/lu.uni.course.bics.global}{{\underline{\textbf{https://github.com/nicolasguelfi/lu.uni.course.bics.global}}}}\\
  
This template is to be used using the Latex document preparation system or using any document preparation system. The whole document should be in between 6000 to 8000 words~\footnote{i.e. approximately 12 to 16 pages double columns excluding the Plagiarism Statement} (excluding the annexes) and the proportions must be preserved. The other documents to be delivered (summaries, \ldots) should have their format adapted from this template.\\

A tutor (or any person having contributed to the BSP work) is not a co-author per se for a student's work. It is possible to exploit a BSP report to produce a scientific and technical publication. In this case, the authors list has to be discussed and agreed with the concerned parties.

\end{abstract}

% no keywords

% For peer review papers, you can put extra information on the cover
% page as needed:
% \ifCLASSOPTIONpeerreview
% \begin{center} \bfseries EDICS Category: 3-BBND \end{center}
% \fi
%
% For peerreview papers, this IEEEtran command inserts a page break and
% creates the second title. It will be ignored for other modes.
\IEEEpeerreviewmaketitle

\section{Plagiarism statement}
{\it This 350 words section without this first paragraph must be included in the submitted report and placed after the conclusion. This section is not counting in the total words quantity.}\\

\newline
I declare that I am aware of the following facts:
\begin{itemize}
	\item As a student at the University of Luxembourg I must respect the rules of intellectual honesty, in particular not to resort to plagiarism, fraud or any other method that is illegal or contrary to scientific integrity.
	\item My report will be checked for plagiarism and if the plagiarism check is positive, an internal procedure will be started by my tutor. I am advised to request a pre-check by my tutor to avoid any issue.
	\item As declared in the assessment procedure of the University of Luxembourg, plagiarism is committed whenever the source of information used in an assignment, research report, paper or otherwise published/circulated piece of work is not properly acknowledged. In other words, plagiarism is the passing off as one’s own the words, ideas or work of another person, without attribution to the author. The omission of such proper acknowledgement amounts to claiming authorship for the work of another person. Plagiarism is committed regardless of the language of the original work used. Plagiarism can be deliberate or accidental.
Instances of plagiarism include, but are not limited to:
\begin{enumerate}
  \item Not putting quotation marks around a quote from another person’s work
  \item Pretending to paraphrase while in fact quoting
  \item Citing incorrectly or incompletely
  \item Failing to cite the source of a quoted or paraphrased work
  \item Copying/reproducing sections of another person’s work without acknowledging the source
  \item Paraphrasing another person’s work without acknowledging the source
  \item Having another person write/author a work for oneself and submitting/publishing it (with permission, with or without compensation) in one’s own name (‘ghost-writing’)
  \item Using another person’s unpublished work without attribution and permission (‘stealing’)
  \item Presenting a piece of work as one’s own that contains a high proportion of quoted/copied or paraphrased text (images, graphs, etc.), even if adequately referenced
\end{enumerate}
Auto- or self-plagiarism, that is the reproduction of (portions of a) text previously written by the author without citing that text, i.e. passing previously authored text as new, may be regarded as fraud if deemed sufficiently severe.
\end{itemize}

\section{Introduction ($\pm$ 5\% of total words)}
% no \IEEEPARstart
This paper presents the bachelor semester project made by Motivated Student together with Motivated Tutor as his motivated tutor.
It presents the scientific and technical dimensions of the work done. All the words written here have been newly created by the authors and if some sequence of words or any graphic information created by others are included then it is explicitly indicated the original reference to the work reused. 

This report separates explicitly the scientific work from the technical one. In deed each BSP must cover those two dimensions with a constrained balance  (cf. \cite{bics-bsp-reference-document}). Thus it is up to the Motivated Tutor and Motivated Student to ensure that the deliverables belonging to each dimension are clearly stated. As an example, a project whose title would be ``PLAYTOUCH - A multi-user game for multi-touch devices'' could define the following deliverables:
\begin{itemize}
  \item Possible scientific~\cite{armstrong2017guidelinesforscience} deliverables:
		\begin{itemize}
			\item What are concurrency models and how are they implemented?
			\item How is measured ergonomics in human-computer interaction?
			\item How to model the concurrency of a multi-touch devices?
			\item Can PLAYTOUCH enter in a blocking state?
			\item How to model the design of PLAYTOUCH?
		\end{itemize}
  \item Possible Technical deliverables:
		\begin{itemize}
			\item PLAYTOUCH Implementation
			\item PLAYTOUCH Tests implementation
			\item Hardware end system configuration for PLAYTOUCH
		\end{itemize}

\end{itemize}
The length of the report should be from 6000 to 8000 words excluding images and annexes. The sections presenting the technical and scientific deliverables represent $\pm$ 80\% of total words of the report.

\section{Project description ($\pm$ 10\% of total words) }
\subsection{Domains}
\subsubsection{Scientific }
Provide a short description of the scientific domain(s) in which the project is being made.
\subsubsection{Technical}
Provide a short description of the technical domain(s) in which the project is being made.

\subsection{Targeted Deliverables}
\label{sec-deliverables}
\subsubsection{Scientific deliverables}
Provide a synthetic and abstract description of the scientific deliverables that have been produced. 
Each BSP must contain some work done according to the principles of the scientific method. It basically means that you should define at least one question related to the knowledge domain of your BSP and follow part of the scientific method process to answer this question. The description of the work done to answer this question is a scientific deliverable.

Other examples of question could be:
\begin{itemize}
	\item Is Python an adequate language for concurrent programs?
	\item How can we measure the ergonomics of a graphical user interface?
	\item How can we ensure that a program will not fail?
\end{itemize}

An answer to such question should be the result of applying partly or totally the scientific method according to its standard definition which can be found in the literature.

As you can see in this template, the scientific deliverable is entirely separated from the technical deliverable. It the default case it addresses a question closely related to the technical deliverable.  

\subsubsection{Technical deliverables}
Provide a synthetic and abstract description of the technical deliverables that were targeted to be produced. A technical deliverable in this report is the description of a product  build by the student using software or hardware technologies.


% \subsection{Constraints}
% Provide all the constraints that were to be considered for the project.
% A constraint is a property that is agreed by you and your PAT to have been satisfied before starting the project.
% An example could be ``good level in Python programming''. 
% As a consequence the work done to satisfy the constraints cannot be presented as a deliverable of the BSP.

\section{Pre-requisites}

\subsection{Scientific pre-requisites}
To successfully understand the scientific report, no actual pre-requisite was necessary.
But something that was useful to comprehend is simple knowledge of comparison.
\subsection{Technical pre-requisites}
The technical part of the BSP is coded in Python, thereby knowledge about coding in Python is recommended.\\
This applies to any general programming knowledge as well not necessarily Python only.\\
Taken in consideration that not all programming languages share same structure or functions.\\


\section{ A Scientific Deliverable 1}
\label{sec-production}
\subsection{Requirements}
The purpose of the deliverable is to research and formalize an answer to the following question:
\emph{How secure is Enigma ?}
Therefore, the points that will be covered are:
\begin{itemize}
    \item{[5.3.1.]} Brief description of Enigma
    \item{[5.3.2.]} Definition of encryption and how it works in the case of Enigma
    \item{[5.3.3.]} Breaking Point N°1: "Key Exchange"
    \item{[5.3.4.]} Breaking Point N°2: "Letters cannot encoded into themselves"
\end{itemize}

\subsection{Design}
The scientific deliverable is going to start off with a brief description of what Enigma is and introduce encryption which will also be further explained in the next point. With that out of the way, the main part of the report is coming up and explain the flaws of Enigma.\\

The first flaw "Key exchange" was of a lesser problem back in the day but to today's standards is as important as the cipher itself. Thereby some examples of today's implementation for key exchange is presented.\\

The second flaw, was more focused on the functionality of how encryption behaves concerning Enigma and explain how this affects the security thereby how it could be used to break it.

\subsection{Production}

\subsubsection{Enigma} Enigma is an encryption device which was mostly used by the Germans during the WWII, it was capable of configuring trillions of combinations and thereby rendering the cipher \underline{almost} impossible to crack and such it was preferred way for German army to communicate.\\
The way it worked it was by permuting each letter into another one with the help of rotors and other components like the plugboard and the reflector. In this case, the rotors act like a step for the letter to move and the rotors change dynamically each time a letter is pressed.

\subsubsection{Encryption} Encryption is another means of securing information with a help of mathematical techniques and algorithms. Usually a key is provided which is used to access the information back. The encryption process offers information unreadable to anybody without the proper key. This technique of securing information is advantageous fighting off hackers who are willing to access sensitive information.

\subsubsection{Key Exchange} Each encryption method is conventionally accompanied by a key that is used to process the information encrypted readable again. And thus, for an encryption method to be reliable security of the key plays a big role. At the beginning, people relied on confidentiality of securing the key and utilized a single key thereby the method was called 'single-key' cryptography.\\
'Single-Key' cryptography in fact provided insecurities on the level where if the two parties failed exchanging properly the key it increased the chance for a third party to gain acquisition of the key.\\
Thereby, another methods have been introduced, specifically a two-key system which was the famous 'public-key' cryptography.\\
It was accompanied with a private and a public key, based on what encryption algorithms was used the uses of these keys depended on it. i.e.: 'Rivest-Shamir-Adleman' (RSA) used the private key for decryption of information where as 'Digital Signature Algorithm' (DSA) used it for authentication purposes.\\
Another method of securely exchanging keys is the famous 'Diffie-Hellman' (DH) method, even though this method was one of the earliest to come to light it's still broadly used on multiple platforms. The method consists of generating am unique public and an unique private key for both parties resulting a combination of one's private key and other's public key gives out the secret information.\\
Enigma on the other hand didn't follow any of these 'modern' methods of key exchange and would simply share list of day-to-day configurations of the Enigma to trusted parties which would for a month. To today's standards, this in fact is very dangerous way of distributing information because if it would land on the wrong hands the leak of information wouldn't be resolved soon enough to prevent drawbacks. (In the case of Enigma, war obstructions)\\ 
Nonetheless, in this case a simple solution is to modernize the key distribution by using more modern methods like stated above (i.e.:  RSA, DSA, DH, \ldots)

\subsubsection{Letters cannot be encoded into themselves}
During the WWII, many of the parties that were trying to break Enigma have noticed a simple but very big flaw in the encryption technique of the machine. Which was too difficult to fix provided by the techniques that existed at the time. The flaw was lying much deeper under the surface and involved many interactions between other components to resolve.
Thereby, another breaking point of Enigma was the fact that a letter cannot represent itself during encryption based on the functionality and how Enigma works.\\
Which means, when the user would press the letter "A", the output would \underline{never} be encoded as the letter "A".\\
When a letter is pressed on the machine, an electrical flow is created which passes through numerous components that help permute the letter, in this case rotors and the reflector.
The electrical flow follows an unique 'pathway' from the pressed letter and lightened up letter. The path never intersects thus the electrical flow can never arrive to the same position where it departed from.\\
This information was a great deal for mathematicians and code-breakers at the time, because it gave them a hint and something to work on because previously they've been working blindly on how to break the Enigma.\\
At this point, since they knew about the flaw they were missing only one element to pursue and that was to guess a word which would always be included into the enciphered messages. In this case, they found out that a weather report and the phrase "Heil Hitler" was always included, where the latter was at the end of a letter and former at the beginning.\\
With these two elements, mathematicians used a specific method to break the code; the method was 'process of elimination'. They would compare words to the letters of the code and in the case where a single letter from word was also in the code, then they were assured that the word is not part of the code based on the fact that a letter cannot be itself.\\

Provided example, let's suppose we're comparing the word "windy" to the code (The symbol \# is to indicate the unknown letters):\\

\begin{tabular}{ |c|c|c|c|c|c|c|c|c| }
 \hline
 Coded Message & \cellcolor{red!35}W & S & I & T & V & N & A \\ 
 \hline
 Word & \cellcolor{red!35}W & \cellcolor{green!35}I & \cellcolor{green!35}N & \cellcolor{green!35}D & \cellcolor{green!35}Y & \# & \# \\
 \hline
\end{tabular}\\

Since the letter 'W' are matching up, means that this doesn't hold up.\\

\begin{tabular}{ |c|c|c|c|c|c|c|c|c| }
 \hline
 Coded Message & W & S & \cellcolor{red!35}I & T & V & N & A \\ 
 \hline
 Word & \# & \cellcolor{green!35}W & \cellcolor{red!35}I & \cellcolor{green!35}N & \cellcolor{green!35}D & \cellcolor{green!35}Y & \# \\
 \hline
\end{tabular}\\

In this case the letter 'I' is matching, thereby also not the right encoding.\\

\begin{tabular}{ |c|c|c|c|c|c|c|c|c| }
 \hline
 Coded Message & W & S & I & T & V & N & A \\ 
 \hline
 Word & \# & \# & \cellcolor{green!35}W & \cellcolor{green!35}I & \cellcolor{green!35}N & \cellcolor{green!35}D & \cellcolor{green!35}Y \\
 \hline
\end{tabular}\\

Finally where none of the letters match up, which means that the encoding works in this case. Take in consideration that the word 'Windy' isn't guaranteed that it is in the secret message itself but it gives us a bigger picture understanding where to start looking for breaking messages.\\


\subsection{Assessment ($\pm$ 15\% of section's words)}
Provide any objective elements to assess that your deliverables do or do not satisfy the requirements described above. 

\section{ A Technical Deliverable 1}
\label{sec-production}

\subsection{Requirements}
The program is supposed to polished and posses similarities but also improve upon if necessary of the original Enigma machine, in this case:
\begin{itemize}
    \item The program should include a relatable interface
    \item Fully configurable rotor settings which would increment dynamically
    \item Unlike the original, a text-box where the user is able to input a whole string of text and process it whole
    \item For the ease of usability, the user should be able to easily test messages. i.e.: an function to import, export and switch output with input
    \item And finally, the program shouldn't depend on anything external to function alike the original
\end{itemize}

\subsection{Design ($\pm$ 30\% of section's words)}
The program is easily readable from the surface as soon as the user is introduced to it; on the top the 'rotor section' is located where the user is supposed to insert his/her set of rotor settings. To confirm the rotors, the user shall press the 'Set' button to finalize the process.\\
This is the first thing the user should be introduced to because this the main functionality of the program and results an unique key for the processed message. The configured settings are important to remember, without them the original message cannot be retrieved easily. Thereby, when the user has set them, the rotors settings are visible to the user thanks to the message above where it clearly states the rotor settings in the form of (X/X/X) where 'X' designates the value of the rotor in the respectful order.\\

Furthermore, the section right below the rotors is where the 'input section' is located. This is where the user is supposed to insert a string of text, that being a simple phrase or a whole paragraph of text, which he/she is wishing to process.\\

Moreover, the similar section right after is where the 'output section' is located. This section is nothing to be mangled with as it only provides an output through the functions of buttons and doesn't provide anything except that.\\

The reason for inclusion of the last two mentioned sections is to provide usability to the user, unlike the original Enigma machine the user was supposed to type each letter through key-by-key method and would result into a annoying pattern where if the user somehow messes up, he's supposed to start all over again. With the inclusion of the 'input section', the user can easily import a whole string of text and process it whole while still following the procedures of dynamic rotors in the back-end and output the whole result in the 'output section' and thereby eliminate any frustration which was known to exist with the original Enigma machine.\\

Finally the given buttons at the end are self-explanatory by their name. i.e.: 'Transfer','Encrypt', 'Decrypt', 'Export*' and 'Import*'\\
A notice is provided to explain that the 'Export*' and 'Import*' buttons function in aid of a text file specifically named 'key.txt' file.\\

The practicality of these buttons extends greater than what it represents on the surface. 'Transfer' button exists due to the reasons if the user is willing to test quickly some length of strings and be provided of what the result may be.\\

'Encrypt' and 'Decrypt' are designated for flexibility of the program as it adds upon options to the user, these in fact are very necessary because they include the main objectives of the program itself. Unlike the original where a mechanical 'pathway' is created each time a key is pressed, that 'pathway' doesn't exist in the case of this program due it to not being necessary to properly function and also in this case the user has options to decide how exact he/she wants it to behave like.\\

The buttons 'Export*' and 'Import*' are again an addition that provide an ease of functionality where the user isn't supposed to write enciphered messages and accompanied rotor configurations manually. The buttons read/create files which include important information (rotor settings and the processed message) and implement them accordingly. In the case of 'Import*', the rotor settings is set and the message is imported into the 'input section' and enables the function to 'Decrypt' the message for the user. On the other hand, the 'Export*' saves the message and settings onto a text-file named 'key.txt'.\\

The 'key.txt' file created upon 'Export' is easily readable by a human but the content cannot be fully understood without the program to decrypt the processed message.\\ 

\subsection{Production ($\pm$ 40\% of section's words)}

These parts are formed into functions, first being the 'rotate character' whose job is to permute a character by a step where the step in this case is a permutation of the rotors.\\

The permutation is a process of 3 different mathematical operations:\\
\begin{itemize}
    \item Firstly adding the rotor 1 to the step;\\
    \item Secondly subtracting the rotor 2 from the step;\\
    \item Finally, adding the double of the rotor 3 to the step.\\
\end{itemize}
Second part is the 'rotor change' whose objective is to increment rotors in a respectful order and make sure that the rotors stay between their boundaries (1-26).\\
Once the first rotor completes a whole turn the second rotor increments by 1 and so on.\\

Third part which in this case is two different functions, those being 'encrypt' and 'decrypt'. The prior being the main function whose objective is to rotate (encrypt) each character while also initiating the 'rotor change' each time. This provides a step for the 'rotate character' which is dynamic in this case.\\

Last part of the program is the functionality of import/export, whose job is create, respectfully read, a text file named 'key.txt'.\\

The intention is to provide usability to the user whilst operating with the program, thereby the change to the interface compared to the original was quite needed to fulfill the intention.\\
This version provides the user to process a whole string of a text instead of the 'key-by-key' method, due to the reason where latter provided a slow process to read longer texts.\\

\subsection{Assessment ($\pm$ 15\% of section's words)}
cf. section~\ref{sec-production} applied to the technical deliverable


\section*{Acknowledgment}
The authors would like to thank the BiCS management and education team for the amazing work done.


\section{Conclusion}
The conclusion goes here.

% An example of a floating figure using the graphicx package.
% Note that \label must occur AFTER (or within) \caption.
% For figures, \caption should occur after the \includegraphics.
% Note that IEEEtran v1.7 and later has special internal code that
% is designed to preserve the operation of \label within \caption
% even when the captionsoff option is in effect. However, because
% of issues like this, it may be the safest practice to put all your
% \label just after \caption rather than within \caption{}.
%
% Reminder: the "draftcls" or "draftclsnofoot", not "draft", class
% option should be used if it is desired that the figures are to be
% displayed while in draft mode.
%
%\begin{figure}[!t]
%\centering
%\includegraphics[width=2.5in]{myfigure}
% where an .eps filename suffix will be assumed under latex, 
% and a .pdf suffix will be assumed for pdflatex; or what has been declared
% via \DeclareGraphicsExtensions.
%\caption{Simulation results for the network.}
%\label{fig_sim}
%\end{figure}

% Note that the IEEE typically puts floats only at the top, even when this
% results in a large percentage of a column being occupied by floats.


% An example of a double column floating figure using two subfigures.
% (The subfig.sty package must be loaded for this to work.)
% The subfigure \label commands are set within each subfloat command,
% and the \label for the overall figure must come after \caption.
% \hfil is used as a separator to get equal spacing.
% Watch out that the combined width of all the subfigures on a 
% line do not exceed the text width or a line break will occur.
%
%\begin{figure*}[!t]
%\centering
%\subfloat[Case I]{\includegraphics[width=2.5in]{box}%
%\label{fig_first_case}}
%\hfil
%\subfloat[Case II]{\includegraphics[width=2.5in]{box}%
%\label{fig_second_case}}
%\caption{Simulation results for the network.}
%\label{fig_sim}
%\end{figure*}
%
% Note that often IEEE papers with subfigures do not employ subfigure
% captions (using the optional argument to \subfloat[]), but instead will
% reference/describe all of them (a), (b), etc., within the main caption.
% Be aware that for subfig.sty to generate the (a), (b), etc., subfigure
% labels, the optional argument to \subfloat must be present. If a
% subcaption is not desired, just leave its contents blank,
% e.g., \subfloat[].


% An example of a floating table. Note that, for IEEE style tables, the
% \caption command should come BEFORE the table and, given that table
% captions serve much like titles, are usually capitalized except for words
% such as a, an, and, as, at, but, by, for, in, nor, of, on, or, the, to
% and up, which are usually not capitalized unless they are the first or
% last word of the caption. Table text will default to \footnotesize as
% the IEEE normally uses this smaller font for tables.
% The \label must come after \caption as always.
%
%\begin{table}[!t]
%% increase table row spacing, adjust to taste
%\renewcommand{\arraystretch}{1.3}
% if using array.sty, it might be a good idea to tweak the value of
% \extrarowheight as needed to properly center the text within the cells
%\caption{An Example of a Table}
%\label{table_example}
%\centering
%% Some packages, such as MDW tools, offer better commands for making tables
%% than the plain LaTeX2e tabular which is used here.
%\begin{tabular}{|c||c|}
%\hline
%One & Two\\
%\hline
%Three & Four\\
%\hline
%\end{tabular}
%\end{table}


% Note that the IEEE does not put floats in the very first column
% - or typically anywhere on the first page for that matter. Also,
% in-text middle ("here") positioning is typically not used, but it
% is allowed and encouraged for Computer Society conferences (but
% not Computer Society journals). Most IEEE journals/conferences use
% top floats exclusively. 
% Note that, LaTeX2e, unlike IEEE journals/conferences, places
% footnotes above bottom floats. This can be corrected via the
% \fnbelowfloat command of the stfloats package.

% trigger a \newpage just before the given reference
% number - used to balance the columns on the last page
% adjust value as needed - may need to be readjusted if
% the document is modified later
%\IEEEtriggeratref{8}
% The "triggered" command can be changed if desired:
%\IEEEtriggercmd{\enlargethispage{-5in}}

% references section

% can use a bibliography generated by BibTeX as a .bbl file
% BibTeX documentation can be easily obtained at:
% http://mirror.ctan.org/biblio/bibtex/contrib/doc/
% The IEEEtran BibTeX style support page is at:
% http://www.michaelshell.org/tex/ieeetran/bibtex/
%\bibliographystyle{IEEEtran}
% argument is your BibTeX string definitions and bibliography database(s)
%\bibliography{IEEEabrv,../bib/paper}
%
% <OR> manually copy in the resultant .bbl file
% set second argument of \begin to the number of references
% (used to reserve space for the reference number labels box)
\begin{thebibliography}{1}

\bibitem[BiCS(2021)]{bics-bsp-report-template}
\newblock {BiCS Bachelor Semester Project Report Template}.
\newblock {https://github.com/nicolasguelfi/lu.uni.course.bics.global}
\newblock {University of Luxembourg, BiCS - Bachelor in Computer Science (2021).}

\bibitem[BiCS(2021)] {bics-bsp-reference-document}
{Bachelor in Computer Science}:
\newblock {BiCS Semester Projects Reference Document}.
\newblock Technical report, University of Luxembourg (2021)

\bibitem[Armstrong and Green(2017)]{armstrong2017guidelinesforscience}
J~Scott Armstrong and Kesten~C Green.
\newblock Guidelines for science: Evidence and checklists.
\newblock \emph{Scholarly Commons}, pages 1--24, 2017.
\newblock {https://repository.upenn.edu/marketing_papers/181/}


\end{thebibliography}
\newpage 
\section{Appendix}
All images and additional material go there.
% that's all folks
\end{document}


