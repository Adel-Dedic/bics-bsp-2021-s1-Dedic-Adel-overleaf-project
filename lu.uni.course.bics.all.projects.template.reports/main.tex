\documentclass[conference,compsoc]{IEEEtran}
 
\usepackage{datetime}

% *** CITATION PACKAGES ***
%
\ifCLASSOPTIONcompsoc
  % IEEE Computer Society needs nocompress option
  % requires cite.sty v4.0 or later (November 2003)
  \usepackage[nocompress]{cite}
\else
  % normal IEEE
  \usepackage{cite} 
\fi 
% cite.sty was written by Donald Arseneau
% V1.6 and later of IEEEtran pre-defines the format of the cite.sty package
% \cite{} output to follow that of the IEEE. Loading the cite package will
% result in citation numbers being automatically sorted and properly
% "compressed/ranged". e.g., [1], [9], [2], [7], [5], [6] without using
% cite.sty will become [1], [2], [5]--[7], [9] using cite.sty. cite.sty's
% \cite will automatically add leading space, if needed. Use cite.sty's
% noadjust option (cite.sty V3.8 and later) if you want to turn this off
% such as if a citation ever needs to be enclosed in parenthesis.
% cite.sty is already installed on most LaTeX systems. Be sure and use
% version 5.0 (2009-03-20) and later if using hyperref.sty.
% The latest version can be obtained at:
% http://www.ctan.org/pkg/cite
% The documentation is contained in the cite.sty file itself.
%
% Note that some packages require special options to format as the Computer
% Society requires. In particular, Computer Society  papers do not use
% compressed citation ranges as is done in typical IEEE papers
% (e.g., [1]-[4]). Instead, they list every citation separately in order
% (e.g., [1], [2], [3], [4]). To get the latter we need to load the cite
% package with the nocompress option which is supported by cite.sty v4.0
% and later.

% *** GRAPHICS RELATED PACKAGES ***
%
\ifCLASSINFOpdf
  % \usepackage[pdftex]{graphicx}
  % declare the path(s) where your graphic files are
  % \graphicspath{{../pdf/}{../jpeg/}}
  % and their extensions so you won't have to specify these with
  % every instance of \includegraphics
  % \DeclareGraphicsExtensions{.pdf,.jpeg,.png}
\else
  % or other class option (dvipsone, dvipdf, if not using dvips). graphicx
  % will default to the driver specified in the system graphics.cfg if no
  % driver is specified.
  % \usepackage[dvips]{graphicx}
  % declare the path(s) where your graphic files are
  % \graphicspath{{../eps/}}
  % and their extensions so you won't have to specify these with
  % every instance of \includegraphics
  % \DeclareGraphicsExtensions{.eps}
\fi
% graphicx was written by David Carlisle and Sebastian Rahtz. It is
% required if you want graphics, photos, etc. graphicx.sty is already
% installed on most LaTeX systems. The latest version and documentation
% can be obtained at: 
% http://www.ctan.org/pkg/graphicx
% Another good source of documentation is "Using Imported Graphics in
% LaTeX2e" by Keith Reckdahl which can be found at:
% http://www.ctan.org/pkg/epslatex
%
% latex, and pdflatex in dvi mode, support graphics in encapsulated
% postscript (.eps) format. pdflatex in pdf mode supports graphics
% in .pdf, .jpeg, .png and .mps (metapost) formats. Users should ensure
% that all non-photo figures use a vector format (.eps, .pdf, .mps) and
% not a bitmapped formats (.jpeg, .png). The IEEE frowns on bitmapped formats
% which can result in "jaggedy"/blurry rendering of lines and letters as
% well as large increases in file sizes.
%
% You can find documentation about the pdfTeX application at:
% http://www.tug.org/applications/pdftex


% *** MATH PACKAGES ***
%
%\usepackage{amsmath}
% A popular package from the American Mathematical Society that provides
% many useful and powerful commands for dealing with mathematics.
%
% Note that the amsmath package sets \interdisplaylinepenalty to 10000
% thus preventing page breaks from occurring within multiline equations. Use:
%\interdisplaylinepenalty=2500
% after loading amsmath to restore such page breaks as IEEEtran.cls normally
% does. amsmath.sty is already installed on most LaTeX systems. The latest
% version and documentation can be obtained at:
% http://www.ctan.org/pkg/amsmath

% *** SPECIALIZED LIST PACKAGES ***
%
%\usepackage{algorithmic}
% algorithmic.sty was written by Peter Williams and Rogerio Brito.
% This package provides an algorithmic environment fo describing algorithms.
% You can use the algorithmic environment in-text or within a figure
% environment to provide for a floating algorithm. Do NOT use the algorithm
% floating environment provided by algorithm.sty (by the same authors) or
% algorithm2e.sty (by Christophe Fiorio) as the IEEE does not use dedicated
% algorithm float types and packages that provide these will not provide
% correct IEEE style captions. The latest version and documentation of
% algorithmic.sty can be obtained at:
% http://www.ctan.org/pkg/algorithms
% Also of interest may be the (relatively newer and more customizable)
% algorithmicx.sty package by Szasz Janos:
% http://www.ctan.org/pkg/algorithmicx


% *** ALIGNMENT PACKAGES ***
%
%\usepackage{array}
% Frank Mittelbach's and David Carlisle's array.sty patches and improves
% the standard LaTeX2e array and tabular environments to provide better
% appearance and additional user controls. As the default LaTeX2e table
% generation code is lacking to the point of almost being broken with
% respect to the quality of the end results, all users are strongly
% advised to use an enhanced (at the very least that provided by array.sty)
% set of table tools. array.sty is already installed on most systems. The
% latest version and documentation can be obtained at:
% http://www.ctan.org/pkg/array

% IEEEtran contains the IEEEeqnarray family of commands that can be used to
% generate multiline equations as well as matrices, tables, etc., of high
% quality.

% *** SUBFIGURE PACKAGES ***
%\ifCLASSOPTIONcompsoc
%  \usepackage[caption=false,font=footnotesize,labelfont=sf,textfont=sf]{subfig}
%\else
%  \usepackage[caption=false,font=footnotesize]{subfig}
%\fi
% subfig.sty, written by Steven Douglas Cochran, is the modern replacement
% for subfigure.sty, the latter of which is no longer maintained and is
% incompatible with some LaTeX packages including fixltx2e. However,
% subfig.sty requires and automatically loads Axel Sommerfeldt's caption.sty
% which will override IEEEtran.cls' handling of captions and this will result
% in non-IEEE style figure/table captions. To prevent this problem, be sure
% and invoke subfig.sty's "caption=false" package option (available since
% subfig.sty version 1.3, 2005/06/28) as this is will preserve IEEEtran.cls
% handling of captions.
% Note that the Computer Society format requires a sans serif font rather
% than the serif font used in traditional IEEE formatting and thus the need
% to invoke different subfig.sty package options depending on whether
% compsoc mode has been enabled.
%
% The latest version and documentation of subfig.sty can be obtained at:
% http://www.ctan.org/pkg/subfig

% *** FLOAT PACKAGES ***
%
%\usepackage{fixltx2e}
% fixltx2e, the successor to the earlier fix2col.sty, was written by
% Frank Mittelbach and David Carlisle. This package corrects a few problems
% in the LaTeX2e kernel, the most notable of which is that in current
% LaTeX2e releases, the ordering of single and double column floats is not
% guaranteed to be preserved. Thus, an unpatched LaTeX2e can allow a
% single column figure to be placed prior to an earlier double column
% figure.
% Be aware that LaTeX2e kernels dated 2015 and later have fixltx2e.sty's
% corrections already built into the system in which case a warning will
% be issued if an attempt is made to load fixltx2e.sty as it is no longer
% needed.
% The latest version and documentation can be found at:
% http://www.ctan.org/pkg/fixltx2e

%\usepackage{stfloats}
% stfloats.sty was written by Sigitas Tolusis. This package gives LaTeX2e
% the ability to do double column floats at the bottom of the page as well
% as the top. (e.g., "\begin{figure*}[!b]" is not normally possible in
% LaTeX2e). It also provides a command:
%\fnbelowfloat
% to enable the placement of footnotes below bottom floats (the standard
% LaTeX2e kernel puts them above bottom floats). This is an invasive package
% which rewrites many portions of the LaTeX2e float routines. It may not work
% with other packages that modify the LaTeX2e float routines. The latest
% version and documentation can be obtained at:
% http://www.ctan.org/pkg/stfloats
% Do not use the stfloats baselinefloat ability as the IEEE does not allow
% \baselineskip to stretch. Authors submitting work to the IEEE should note
% that the IEEE rarely uses double column equations and that authors should try
% to avoid such use. Do not be tempted to use the cuted.sty or midfloat.sty
% packages (also by Sigitas Tolusis) as the IEEE does not format its papers in
% such ways.
% Do not attempt to use stfloats with fixltx2e as they are incompatible.
% Instead, use Morten Hogholm'a dblfloatfix which combines the features
% of both fixltx2e and stfloats:
%
% \usepackage{dblfloatfix}
% The latest version can be found at:
% http://www.ctan.org/pkg/dblfloatfix

% *** PDF, URL AND HYPERLINK PACKAGES ***
%
%\usepackage{url}
% url.sty was written by Donald Arseneau. It provides better support for
% handling and breaking URLs. url.sty is already installed on most LaTeX
% systems. The latest version and documentation can be obtained at:
% http://www.ctan.org/pkg/url
% Basically, \url{my_url_here}.

% *** Do not adjust lengths that control margins, column widths, etc. ***
% *** Do not use packages that alter fonts (such as pslatex).         ***
% There should be no need to do such things with IEEEtran.cls V1.6 and later.
% (Unless specifically asked to do so by the journal or conference you plan
% to submit to, of course. )

% correct bad hyphenation here
\hyphenation{op-tical net-works semi-conduc-tor}
   
\usepackage{hyperref}
 
\begin{document}
% 
% paper title
% Titles are generally capitalized except for words such as a, an, and, as,
% at, but, by, for, in, nor, of, on, or, the, to and up, which are usually
% not capitalized unless they are the first or last word of the title.
% Linebreaks \\ can be used within to get better formatting as desired.
% Do not put math or special symbols in the title.
\title{BSPro - A First Bachelor Semester Project in BiCS-land\\
{\small \today~-~\currenttime}}

 
% author names and affiliations
% use a multiple column layout for up to three different
% affiliations
\author{\IEEEauthorblockN{Dedic Adel}
\IEEEauthorblockA{University of Luxembourg\\
Email: adel.dedic.001@student.uni.lu}
\\
{\bf This report has been produced under the supervision of:}\\
\IEEEauthorblockN{Benoit Ries}
\IEEEauthorblockA{University of Luxembourg\\
Email: benoit.ries@uni.lu}%
}

% conference papers do not typically use \thanks and this command
% is locked out in conference mode. If really needed, such as for
% the acknowledgment of grants, issue a \IEEEoverridecommandlockouts
% after \documentclass

% for over three affiliations, or if they all won't fit within the width
% of the page (and note that there is less available width in this regard for
% compsoc conferences compared to traditional conferences), use this
% alternative format:
% 
%\author{\IEEEauthorblockN{Michael Shell\IEEEauthorrefmark{1},
%Homer Simpson\IEEEauthorrefmark{2},
%James Kirk\IEEEauthorrefmark{3}, 
%Montgomery Scott\IEEEauthorrefmark{3} and
%Eldon Tyrell\IEEEauthorrefmark{4}}
%\IEEEauthorblockA{\IEEEauthorrefmark{1}School of Electrical and Computer Engineering\\
%Georgia Institute of Technology,
%Atlanta, Georgia 30332--0250\\ Email: see http://www.michaelshell.org/contact.html}
%\IEEEauthorblockA{\IEEEauthorrefmark{2}Twentieth Century Fox, Springfield, USA\\
%Email: homer@thesimpsons.com}
%\IEEEauthorblockA{\IEEEauthorrefmark{3}Starfleet Academy, San Francisco, California 96678-2391\\
%Telephone: (800) 555--1212, Fax: (888) 555--1212}
%\IEEEauthorblockA{\IEEEauthorrefmark{4}Tyrell Inc., 123 Replicant Street, Los Angeles, California 90210--4321}}




% use for special paper notices
%\IEEEspecialpapernotice{(Invited Paper)}




% make the title area
\maketitle

%to remove for your report
%\footnote{}

% As a general rule, do not put math, special symbols or citations
% in the abstract
\begin{abstract}
This document is a template for the scientific and technical (S\&T for short) report that is to be delivered by any BiCS student at the end of each Bachelor Semester Project (BSP). The Latex source files are available at:\\
\href{https://github.com/nicolasguelfi/lu.uni.course.bics.global}{{\underline{\textbf{https://github.com/nicolasguelfi/lu.uni.course.bics.global}}}}\\
  
This template is to be used using the Latex document preparation system or using any document preparation system. The whole document should be in between 6000 to 8000 words~\footnote{i.e. approximately 12 to 16 pages double columns excluding the Plagiarism Statement} (excluding the annexes) and the proportions must be preserved. The other documents to be delivered (summaries, \ldots) should have their format adapted from this template.\\

A tutor (or any person having contributed to the BSP work) is not a co-author per se for a student's work. It is possible to exploit a BSP report to produce a scientific and technical publication. In this case, the authors list has to be discussed and agreed with the concerned parties.

\end{abstract}

% no keywords

% For peer review papers, you can put extra information on the cover
% page as needed:
% \ifCLASSOPTIONpeerreview
% \begin{center} \bfseries EDICS Category: 3-BBND \end{center}
% \fi
%
% For peerreview papers, this IEEEtran command inserts a page break and
% creates the second title. It will be ignored for other modes.
\IEEEpeerreviewmaketitle

\section{Plagiarism statement}
{\it This 350 words section without this first paragraph must be included in the submitted report and placed after the conclusion. This section is not counting in the total words quantity.}\\

\newline
I declare that I am aware of the following facts:
\begin{itemize}
	\item As a student at the University of Luxembourg I must respect the rules of intellectual honesty, in particular not to resort to plagiarism, fraud or any other method that is illegal or contrary to scientific integrity.
	\item My report will be checked for plagiarism and if the plagiarism check is positive, an internal procedure will be started by my tutor. I am advised to request a pre-check by my tutor to avoid any issue.
	\item As declared in the assessment procedure of the University of Luxembourg, plagiarism is committed whenever the source of information used in an assignment, research report, paper or otherwise published/circulated piece of work is not properly acknowledged. In other words, plagiarism is the passing off as one’s own the words, ideas or work of another person, without attribution to the author. The omission of such proper acknowledgement amounts to claiming authorship for the work of another person. Plagiarism is committed regardless of the language of the original work used. Plagiarism can be deliberate or accidental.
Instances of plagiarism include, but are not limited to:
\begin{enumerate}
  \item Not putting quotation marks around a quote from another person’s work
  \item Pretending to paraphrase while in fact quoting
  \item Citing incorrectly or incompletely
  \item Failing to cite the source of a quoted or paraphrased work
  \item Copying/reproducing sections of another person’s work without acknowledging the source
  \item Paraphrasing another person’s work without acknowledging the source
  \item Having another person write/author a work for oneself and submitting/publishing it (with permission, with or without compensation) in one’s own name (‘ghost-writing’)
  \item Using another person’s unpublished work without attribution and permission (‘stealing’)
  \item Presenting a piece of work as one’s own that contains a high proportion of quoted/copied or paraphrased text (images, graphs, etc.), even if adequately referenced
\end{enumerate}
Auto- or self-plagiarism, that is the reproduction of (portions of a) text previously written by the author without citing that text, i.e. passing previously authored text as new, may be regarded as fraud if deemed sufficiently severe.
\end{itemize}

\section{Introduction ($\pm$ 5\% of total words)}
% no \IEEEPARstart
This paper presents the bachelor semester project made by Motivated Student together with Motivated Tutor as his motivated tutor.
It presents the scientific and technical dimensions of the work done. All the words written here have been newly created by the authors and if some sequence of words or any graphic information created by others are included then it is explicitly indicated the original reference to the work reused. 

This report separates explicitly the scientific work from the technical one. In deed each BSP must cover those two dimensions with a constrained balance  (cf. \cite{bics-bsp-reference-document}). Thus it is up to the Motivated Tutor and Motivated Student to ensure that the deliverables belonging to each dimension are clearly stated. As an example, a project whose title would be ``PLAYTOUCH - A multi-user game for multi-touch devices'' could define the following deliverables:
\begin{itemize}
  \item Possible scientific~\cite{armstrong2017guidelinesforscience} deliverables:
		\begin{itemize}
			\item What are concurrency models and how are they implemented?
			\item How is measured ergonomics in human-computer interaction?
			\item How to model the concurrency of a multi-touch devices?
			\item Can PLAYTOUCH enter in a blocking state?
			\item How to model the design of PLAYTOUCH?
		\end{itemize}
  \item Possible Technical deliverables:
		\begin{itemize}
			\item PLAYTOUCH Implementation
			\item PLAYTOUCH Tests implementation
			\item Hardware end system configuration for PLAYTOUCH
		\end{itemize}

\end{itemize}
The length of the report should be from 6000 to 8000 words excluding images and annexes. The sections presenting the technical and scientific deliverables represent $\pm$ 80\% of total words of the report.

\section{Project description ($\pm$ 10\% of total words) }
\subsection{Domains}
\subsubsection{Scientific }
Provide a short description of the scientific domain(s) in which the project is being made.
\subsubsection{Technical}
Provide a short description of the technical domain(s) in which the project is being made.

\subsection{Targeted Deliverables}
\label{sec-deliverables}
\subsubsection{Scientific deliverables}
Provide a synthetic and abstract description of the scientific deliverables that have been produced. 
Each BSP must contain some work done according to the principles of the scientific method. It basically means that you should define at least one question related to the knowledge domain of your BSP and follow part of the scientific method process to answer this question. The description of the work done to answer this question is a scientific deliverable.

Other examples of question could be:
\begin{itemize}
	\item Is Python an adequate language for concurrent programs?
	\item How can we measure the ergonomics of a graphical user interface?
	\item How can we ensure that a program will not fail?
\end{itemize}

An answer to such question should be the result of applying partly or totally the scientific method according to its standard definition which can be found in the literature.

As you can see in this template, the scientific deliverable is entirely separated from the technical deliverable. It the default case it addresses a question closely related to the technical deliverable.  

\subsubsection{Technical deliverables}
Provide a synthetic and abstract description of the technical deliverables that were targeted to be produced. A technical deliverable in this report is the description of a product  build by the student using software or hardware technologies.


% \subsection{Constraints}
% Provide all the constraints that were to be considered for the project.
% A constraint is a property that is agreed by you and your PAT to have been satisfied before starting the project.
% An example could be ``good level in Python programming''. 
% As a consequence the work done to satisfy the constraints cannot be presented as a deliverable of the BSP.

\section{Pre-requisites ($[5\%..10\%]$ of total words)} 
Describe in these sections the main scientific and technical knowledge that is required to be known by you before starting the project.
Do not describe in details this knowledge but only abstractly. All the content of this section shall not be used, even partially, in the deliverable sections.
It is important not to include in this section all the knowledge you have been obliged to acquire in order to produce the deliverable. It should only state the knowledge the student possessed before starting the project and that was mandatory to possess to be capable to produce the deliverables. It explicitly defines  the technical and scientific pre-condition for the project. It is also useful to avoid project failures due to over or under complex subjects.


\subsection{Scientific pre-requisites}
\subsection{Technical pre-requisites}


\section{ A Scientific Deliverable 1}
For each scientific deliverable targeted in section~\ref{sec-deliverables} provide a full section with all the subsections described below.
\label{sec-production}
\subsection{Requirements ($\pm$ 15\% of section's words)}
Describe here all the properties that characterize the deliverables you produced. It should describe, for each main deliverable, what are the expected functional and non functional properties of the deliverables, who are the actors exploiting the deliverables. It is expected that you have at least one scientific deliverable (e.g. ``Scientific presentation of the Python programming language'', ``State of the art on quality models for human computer interaction'', \ldots.) and one technical deliverable (e.g. ``BSProSoft - A python/django web-site for IT job offers retrieval and analysis'', \ldots). 
\subsection{Design ($\pm$ 30\% of section's words)}
Provide the necessary and most useful explanations on how those deliverables have been produced.
\subsection{Production ($\pm$ 40\% of section's words)}
Provide descriptions of the deliverables concrete production. It must present part of the deliverable (e.g. source code extracts, scientific work extracts, \ldots) to illustrate and explain its actual production.
\subsection{Assessment ($\pm$ 15\% of section's words)}
Provide any objective elements to assess that your deliverables do or do not satisfy the requirements described above. 

\section{ A Technical Deliverable 1}
For each technical deliverable targeted in section~\ref{sec-deliverables} provide a full section with all the subsections described below.
The cumulative volume of all deliverable sections represents 75\% of the paper's volume in words. Volumes below are indicated relative the the section.
\label{sec-production}
\subsection{Requirements ($\pm$ 15\% of section's words)}
cf. section~\ref{sec-production} applied to the technical deliverable
\subsection{Design ($\pm$ 30\% of section's words)}
cf. section~\ref{sec-production} applied to the technical deliverable
\subsection{Production ($\pm$ 40\% of section's words)}
cf. section~\ref{sec-production} applied to the technical deliverable
\subsection{Assessment ($\pm$ 15\% of section's words)}
cf. section~\ref{sec-production} applied to the technical deliverable


\section*{Acknowledgment}
The authors would like to thank the BiCS management and education team for the amazing work done.


\section{Conclusion}
The conclusion goes here.

% An example of a floating figure using the graphicx package.
% Note that \label must occur AFTER (or within) \caption.
% For figures, \caption should occur after the \includegraphics.
% Note that IEEEtran v1.7 and later has special internal code that
% is designed to preserve the operation of \label within \caption
% even when the captionsoff option is in effect. However, because
% of issues like this, it may be the safest practice to put all your
% \label just after \caption rather than within \caption{}.
%
% Reminder: the "draftcls" or "draftclsnofoot", not "draft", class
% option should be used if it is desired that the figures are to be
% displayed while in draft mode.
%
%\begin{figure}[!t]
%\centering
%\includegraphics[width=2.5in]{myfigure}
% where an .eps filename suffix will be assumed under latex, 
% and a .pdf suffix will be assumed for pdflatex; or what has been declared
% via \DeclareGraphicsExtensions.
%\caption{Simulation results for the network.}
%\label{fig_sim}
%\end{figure}

% Note that the IEEE typically puts floats only at the top, even when this
% results in a large percentage of a column being occupied by floats.


% An example of a double column floating figure using two subfigures.
% (The subfig.sty package must be loaded for this to work.)
% The subfigure \label commands are set within each subfloat command,
% and the \label for the overall figure must come after \caption.
% \hfil is used as a separator to get equal spacing.
% Watch out that the combined width of all the subfigures on a 
% line do not exceed the text width or a line break will occur.
%
%\begin{figure*}[!t]
%\centering
%\subfloat[Case I]{\includegraphics[width=2.5in]{box}%
%\label{fig_first_case}}
%\hfil
%\subfloat[Case II]{\includegraphics[width=2.5in]{box}%
%\label{fig_second_case}}
%\caption{Simulation results for the network.}
%\label{fig_sim}
%\end{figure*}
%
% Note that often IEEE papers with subfigures do not employ subfigure
% captions (using the optional argument to \subfloat[]), but instead will
% reference/describe all of them (a), (b), etc., within the main caption.
% Be aware that for subfig.sty to generate the (a), (b), etc., subfigure
% labels, the optional argument to \subfloat must be present. If a
% subcaption is not desired, just leave its contents blank,
% e.g., \subfloat[].


% An example of a floating table. Note that, for IEEE style tables, the
% \caption command should come BEFORE the table and, given that table
% captions serve much like titles, are usually capitalized except for words
% such as a, an, and, as, at, but, by, for, in, nor, of, on, or, the, to
% and up, which are usually not capitalized unless they are the first or
% last word of the caption. Table text will default to \footnotesize as
% the IEEE normally uses this smaller font for tables.
% The \label must come after \caption as always.
%
%\begin{table}[!t]
%% increase table row spacing, adjust to taste
%\renewcommand{\arraystretch}{1.3}
% if using array.sty, it might be a good idea to tweak the value of
% \extrarowheight as needed to properly center the text within the cells
%\caption{An Example of a Table}
%\label{table_example}
%\centering
%% Some packages, such as MDW tools, offer better commands for making tables
%% than the plain LaTeX2e tabular which is used here.
%\begin{tabular}{|c||c|}
%\hline
%One & Two\\
%\hline
%Three & Four\\
%\hline
%\end{tabular}
%\end{table}


% Note that the IEEE does not put floats in the very first column
% - or typically anywhere on the first page for that matter. Also,
% in-text middle ("here") positioning is typically not used, but it
% is allowed and encouraged for Computer Society conferences (but
% not Computer Society journals). Most IEEE journals/conferences use
% top floats exclusively. 
% Note that, LaTeX2e, unlike IEEE journals/conferences, places
% footnotes above bottom floats. This can be corrected via the
% \fnbelowfloat command of the stfloats package.

% trigger a \newpage just before the given reference
% number - used to balance the columns on the last page
% adjust value as needed - may need to be readjusted if
% the document is modified later
%\IEEEtriggeratref{8}
% The "triggered" command can be changed if desired:
%\IEEEtriggercmd{\enlargethispage{-5in}}

% references section

% can use a bibliography generated by BibTeX as a .bbl file
% BibTeX documentation can be easily obtained at:
% http://mirror.ctan.org/biblio/bibtex/contrib/doc/
% The IEEEtran BibTeX style support page is at:
% http://www.michaelshell.org/tex/ieeetran/bibtex/
%\bibliographystyle{IEEEtran}
% argument is your BibTeX string definitions and bibliography database(s)
%\bibliography{IEEEabrv,../bib/paper}
%
% <OR> manually copy in the resultant .bbl file
% set second argument of \begin to the number of references
% (used to reserve space for the reference number labels box)
\begin{thebibliography}{1}

\bibitem[BiCS(2021)]{bics-bsp-report-template}
\newblock {BiCS Bachelor Semester Project Report Template}.
\newblock {https://github.com/nicolasguelfi/lu.uni.course.bics.global}
\newblock {University of Luxembourg, BiCS - Bachelor in Computer Science (2021).}

\bibitem[BiCS(2021)] {bics-bsp-reference-document}
{Bachelor in Computer Science}:
\newblock {BiCS Semester Projects Reference Document}.
\newblock Technical report, University of Luxembourg (2021)

\bibitem[Armstrong and Green(2017)]{armstrong2017guidelinesforscience}
J~Scott Armstrong and Kesten~C Green.
\newblock Guidelines for science: Evidence and checklists.
\newblock \emph{Scholarly Commons}, pages 1--24, 2017.
\newblock {https://repository.upenn.edu/marketing_papers/181/}


\end{thebibliography}
\newpage 
\section{Appendix}
All images and additional material go there.
% that's all folks
\end{document}


